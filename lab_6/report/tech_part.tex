\chapter{Технологическая часть}

В данном разделе будут рассмотрены средства, использованные для разработки, а также приведены коды реализованных алгоритмов.

\section{Средства реализации}

Python был выбран, потому что он предлагает все необходимые инструменты для реализации алгоритмов.


\subsection{Алгоритм поиска решения задачи коммивояжера полным перебором}

В листинге~\ref{lst:full_check} представлен алгоритм поиска решения задачи коммивояжера полным перебором.
\begin{center}
\begin{lstlisting}[caption={Алгоритм поиска решения задачи коммивояжера полным перебором}, label={lst:full_check}]
def full_check(distance_matrix):
    n = len(distance_matrix)
    cities = list(range(n))
    best_length = float('inf')

    perms = generate_permutations(cities)
    for perm in perms:
        cycle = perm + [perm[0]]

        length = 0
        for i in range(n):
            from_city = cycle[i]
            to_city = cycle[i + 1]
            distance = distance_matrix[from_city][to_city]
            length += distance

        if length < best_length:
            best_length = length

    return best_length
\end{lstlisting}
\end{center}

В листинге~\ref{lst:generate_permutations} представлен алгоритм генерации всех возможных перестановки элементов входного списка.
\begin{center}
\begin{lstlisting}[caption={Алгоритм генерации всех возможных перестановки элементов входного списка}, label={lst:generate_permutations}]
def generate_permutations(elements):
    n = len(elements)
    if n == 0:
        return [[]]

    permutations = []
    for i in range(n):
        generated_permutations = generate_permutations(elements[:i] + elements[i+1:])
        for perm in generated_permutations:
            permutations.append([elements[i]] + perm)
    return permutations
\end{lstlisting}
\end{center}

\subsection{Алгоритм поиска решения задачи коммивояжера муравьиным алгоритмом}

В листинге~\ref{lst:ant_colony_optimization} представлен алгоритм поиска решения задачи коммивояжера муравьиным алгоритмом.
\begin{center}
\begin{lstlisting}[caption={Алгоритм поиска решения задачи коммивояжера муравьиным алгоритмом}, label={lst:ant_colony_optimization}]
def ant_colony_optimization(n, distance_matrix, alpha, beta, rho, Q, tmax, n_ants):
    visibility = initialize_visibility(distance_matrix)
    pheromones = initialize_pheromones(n)

    best_route = None
    best_length = float('inf')

    for t in range(tmax):
        routes = []
        lengths = []

        for ant in range(n_ants):
            current_city = random.randint(0, n - 1)
            visited = [False] * n
            visited[current_city] = True
            route = [current_city]

            while len(route) < n:
                probabilities = calculate_probabilities(pheromones, visibility, visited, current_city, alpha, beta)
                r = random.random()
                cumulative = 0
                next_city = None
                for j, prob in enumerate(probabilities):
                    cumulative += prob
                    if r <= cumulative + 1e-5:
                        next_city = j
                        break

                route.append(next_city)
                visited[next_city] = True
                current_city = next_city

            route.append(route[0])
            routes.append(route)

            length = calculate_route_length(route, distance_matrix)
            lengths.append(length)

        min_length = min(lengths)
        if min_length < best_length:
            best_length = min_length
            best_route = routes[lengths.index(min_length)]

        pheromones = update_pheromones(distance_matrix, pheromones, routes, lengths, rho, Q, elite_route=best_route)

    return best_route, best_length
\end{lstlisting}
\end{center}

В листинге~\ref{lst:initialize_visibility} представлен алгоритм инициализации матрицы видимости.
\begin{center}
\begin{lstlisting}[caption={Алгоритм инициализации матрицы видимости}, label={lst:initialize_visibility}]
def initialize_visibility(distance_matrix):
    n = len(distance_matrix)
    visibility = []

    for i in range(n):
        row = []
        for j in range(n):
            if distance_matrix[i][j] > 0:
                visibility_value = 1 / distance_matrix[i][j]
            else:
                visibility_value = 0
            row.append(visibility_value)
        visibility.append(row)
    return visibility
\end{lstlisting}
\end{center}

В листинге~\ref{lst:calculate_probabilities} представлен алгоритм вычисления вероятности перехода из текущего города в другие города.
\begin{center}
\begin{lstlisting}[caption={Алгоритм вычисления вероятности перехода из текущего города в другие города}, label={lst:calculate_probabilities}]
def calculate_probabilities(pheromones, visibility, visited, current_city, alpha, beta):
    n = len(pheromones)
    probabilities = [0] * n
    for j in range(n):
        if not visited[j]:
            probabilities[j] = (pheromones[current_city][j] ** alpha) * (visibility[current_city][j] ** beta)
    total = sum(probabilities)
    if total == 0:
        probabilities = [1 if not visited[j] else 0 for j in range(n)]
        total = sum(probabilities)
    return [p / total for p in probabilities]
\end{lstlisting}
\end{center}

В листинге~\ref{lst:calculate_route_length} представлен алгоритм вычисления длины маршрута.
\begin{center}
\begin{lstlisting}[caption={Алгоритм вычисления вычисления длины маршрута}, label={lst:calculate_route_length}]
def calculate_route_length(route, distance_matrix):
    length = 0
    for i in range(len(route) - 1):
        from_city = route[i]
        to_city = route[i + 1]
        distance = distance_matrix[from_city][to_city]
        length += distance
    return length
\end{lstlisting}
\end{center}

В листинге~\ref{lst:update_pheromones} представлен алгоритм обновления феромонов.
\begin{center}
\begin{lstlisting}[caption={Алгоритм обновления феромонов}, label={lst:update_pheromones}]
def update_pheromones(distance_matrix, pheromones, routes, lengths, rho, Q, elite_route=None):
    n = len(pheromones)
    for i in range(n):
        for j in range(n):
            pheromones[i][j] *= (1 - rho)

    for route, length in zip(routes, lengths):
        for i in range(len(route) - 1):
            pheromones[route[i]][route[i + 1]] += Q / length

    if elite_route:
        elite_length = sum(distance_matrix[elite_route[i]][elite_route[i + 1]] for i in range(len(elite_route) - 1))
        for i in range(len(elite_route) - 1):
            pheromones[elite_route[i]][elite_route[i + 1]] += Q / elite_length

    return pheromones
\end{lstlisting}
\end{center}

\section{Тестовые данные}

\subsection*{Используемые матрицы расстояний}

1. \textbf{Матрица 1×1}
\[
\text{distance\_matrix\_1} = 
\begin{bmatrix}
0
\end{bmatrix}
\]

2. \textbf{Матрица 3×3}
\[
\text{distance\_matrix\_3} = 
\begin{bmatrix}
0 & 10 & 15 \\
10 & 0 & 20 \\
15 & 20 & 0
\end{bmatrix}
\]

3. \textbf{Матрица 5×5}
\[
\text{distance\_matrix\_5} = 
\begin{bmatrix}
0 & 12 & 10 & 29 & 15 \\
12 & 0 & 8 & 30 & 18 \\
10 & 8 & 0 & 25 & 16 \\
29 & 30 & 25 & 0 & 22 \\
15 & 18 & 16 & 22 & 0
\end{bmatrix}
\]

\subsection*{Результаты тестирования}

Результаты тестирования приведены в таблице~\ref{tab:results}.


\begin{longtable}{|c|c|c|c|c|p{4cm}|}
\caption{Результаты тестирования\label{tab:results}} \\ \hline
\textbf{№ теста} & \textbf{Размер матрицы} & \textbf{Алгоритм} & \textbf{Ожидаемый рез-т} & \textbf{Фактический рез-т} \\ \hline
1 & 1×1 & Полный перебор & 0, [0] & 0, [0]  \\ \hline
2 & 1×1 & Муравьиный алгоритм & 0, [0] & 0, [0]  \\ \hline
3 & 3×3 & Полный перебор & 45, [0, 1, 2, 0] & 45, [0, 1, 2, 0] \\ \hline
4 & 3×3 & Муравьиный алгоритм & 45, [0, 1, 2, 0] & 45, [0, 1, 2, 0] \\ \hline
5 & 5×5 & Полный перебор & 88, [0, 2, 1, 4, 3, 0] & 88, [0, 2, 1, 4, 3, 0] \\ \hline
6 & 5×5 & Муравьиный алгоритм & 88, [0, 2, 1, 4, 3, 0] & 88, [0, 2, 1, 4, 3, 0] \\ \hline
\end{longtable}

Все тесты пройдены успешно.

\section*{Вывод}

Были разработаны и протестированы реализованные алгоритмы.


