\chapter{Исследовательская часть}

Технические характеристики устройства:
\begin{itemize}
    \item процессор: Intel(R) Core(TM) i5-10300H с тактовой частотой 2500000000 герц;
	\item ядра:	4;
	\item логические процессоры:	8;
    \item оперативная система: Windows 10;
    \item оперативная память: 8 ГБ.
\end{itemize}

Параметризация была выполнена для одного набора данных, состоящего из трех матриц смежности размера 9x9. Матрицы состоят из координат городов древнего мира. Используемые города приведены в таблицах~\ref{tab:cities_1}, ~\ref{tab:cities_2} и ~\ref{tab:cities_3}.

\begin{table}[H]
\centering
\caption{Таблица городов 1\label{tab:cities_1}}
\begin{tabular}{|l|c|c|}
\hline
\textbf{Название}         & \textbf{Широта} & \textbf{Долгота} \\ \hline
Вавилон                   & 32.536          & 44.420           \\ \hline
Мемфис                    & 29.849          & 31.254           \\ \hline
Фивы (Египет)             & 25.718          & 32.645           \\ \hline
Иерусалим                 & 31.768          & 35.213           \\ \hline
Афины                     & 37.984          & 23.728           \\ \hline
Дамаск                    & 33.513          & 36.292           \\ \hline
Урарту                    & 38.501          & 43.370           \\ \hline
Мохенджо-Даро             & 27.327          & 68.132           \\ \hline
Харран                    & 36.867          & 39.031           \\ \hline
\end{tabular}
\end{table}


\begin{table}[H]
\centering
\caption{Таблица городов 2\label{tab:cities_2}}
\begin{tabular}{|l|c|c|}
\hline
\textbf{Название}         & \textbf{Широта} & \textbf{Долгота} \\ \hline
Рим                       & 41.902          & 12.496           \\ \hline
Карфаген                  & 36.853          & 10.323           \\ \hline
Александрия               & 31.200          & 29.918           \\ \hline
Персеполь                 & 29.934          & 52.891           \\ \hline
Ниневия                   & 36.360          & 43.150           \\ \hline
Троя                      & 39.957          & 26.238           \\ \hline
Спарта                    & 37.075          & 22.429           \\ \hline
Сузиана                   & 32.194          & 48.247           \\ \hline
Кносс                     & 35.299          & 25.162           \\ \hline
\end{tabular}
\end{table}

\begin{table}[H]
\centering
\caption{Таблица городов 3\label{tab:cities_3}}
\begin{tabular}{|l|c|c|}
\hline
\textbf{Название}         & \textbf{Широта} & \textbf{Долгота} \\ \hline
Эфес                      & 37.941          & 27.341           \\ \hline
Тир                       & 33.273          & 35.196           \\ \hline
Коринф                    & 37.905          & 22.934           \\ \hline
Пальмира                  & 34.551          & 38.276           \\ \hline
Угарит                    & 35.595          & 35.782           \\ \hline
Сидон                     & 33.561          & 35.375           \\ \hline
Гиза                      & 29.978          & 31.134           \\ \hline
Луксор                    & 25.687          & 32.639           \\ \hline
Тартесс                   & 37.191          & -6.929           \\ \hline
\end{tabular}
\end{table}

Результаты параметризации приведены в таблице приложения А. Таблица~\ref{tab:experiment_results} содержит результаты параметризации для оптимальных параметров.

\begin{table}[H]
%\small
\centering
\caption{Результаты исследования}
\begin{tabular}{|c|c|c|c|c|c|c|c|c|c|c|c|c|}
\hline
\multicolumn{3}{|c|}{} & \multicolumn{3}{|c|}{Граф 1} & \multicolumn{3}{|c|}{Граф 2} & \multicolumn{3}{|c|}{Граф 3} \\ \hline
\textbf{\boldmath$\alpha$} & \textbf{\boldmath$p$} & \textbf{tmax} & \textbf{max} & \textbf{med} & \textbf{avg} & \textbf{max} & \textbf{med} & \textbf{avg} & \textbf{max} & \textbf{med} & \textbf{avg} \\ \hline
0.5 & 0.5 & 150 & 177 & 0 & 35 & 0 & 0 & 0 & 0 & 0 & 0 \\ \hline
0.5 & 0.75 & 200 & 177 & 0 & 17 & 0 & 0 & 0 & 0 & 0 & 0 \\ \hline
0.5 & 0.9 & 200 & 177 & 0 & 35 & 0 & 0 & 0 & 0 & 0 & 0 \\ \hline
0.75 & 0.5 & 200 & 296 & 0 & 70 & 0 & 0 & 0 & 0 & 0 & 0 \\ \hline
0.75 & 0.9 & 150 & 234 & 0 & 41 & 0 & 0 & 0 & 0 & 0 & 0 \\ \hline
0.9 & 0.9 & 200 & 216 & 0 & 39 & 0 & 0 & 0 & 5 & 0 & 0 \\ \hline
\end{tabular}
\label{tab:experiment_results}
\end{table}



\section*{Вывод}

В результате исследовательской части было проведено исследование производительности алгоритма на различных наборах параметров.
Параметризация алгоритма позволила выделить оптимальные настройки. Лучшими значениями параметров для достижения высокой точности являются: $\alpha = 0.5$, $p = 0.75$, и максимальное количество итераций $t_{\text{max}} = 200$, что подтверждается данными из таблицы~\ref{tab:experiment_results}. 

\clearpage
