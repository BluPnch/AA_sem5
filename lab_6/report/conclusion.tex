\ssr{ЗАКЛЮЧЕНИЕ}

\vspace{0.5cm}
Цель лабораторной работы была достигнута — исследованы методы решения задачи коммивояжера, а также выполнен анализ их преимуществ и недостатков. 

\vspace{0.5cm}
В ходе работы были выполнены следующие задачи:
\begin{enumerate}
    \item исследован и реализован метод полного перебора;
    \item описана математическая основа муравьиного алгоритма;
    \item разработан и реализован муравьиный алгоритм с учетом особенностей перехода между городами;
    \item проведена параметризация муравьиного алгоритма по параметрам $\alpha$, $p$, и числу итераций, результаты оформлены в таблицы и графики;
\end{enumerate}

\vspace{0.5cm}
В результате проведенного исследования установлено, что метод полного перебора обеспечивает точное решение задачи коммивояжера, однако его применение ограничено из-за высокой вычислительной сложности. Муравьиный алгоритм, напротив, демонстрирует высокую эффективность на больших графах при условии правильной настройки параметров. Оптимальные параметры алгоритма были установлены путем расчетов и анализа.

