\chapter{Аналитическая часть}

\section{Задача коммивояжера}

Задача коммивояжера является классической задачей оптимизации и комбинаторики и заключается в нахождении оптимального маршрута, который проходит через все указанные пункты (города) хотя бы по одному разу с последующим возвращением в исходный пункт (город)~\cite{lit1}.

В общем случае задача формулируется на графе, где вершины соответствуют городам, а ребра (или дуги в ориентированном графе) имеют вес, определяющий расстояние, необходимое для перемещения между двумя городами. Цель состоит в минимизации общей длины (затрат) маршрута.
Задача коммивояжера принадлежит к числу NP-трудных задач, что означает, что для ее точного решения в общем случае требуется экспоненциальное время при увеличении количества городов. Требуемое для решения задачи время пропорционально \((n-1)!\), где \(n\) --- количество пунктов. Это делает применение переборных методов непрактичным для больших графов. Например, для задачи с числом городов более 50 нахождение оптимального маршрута потребовало бы вычислительной мощности компьютеров всего мира~\cite{lit1}.

\section{Полный перебор}

Метод полного перебора подразумевает последовательный перебор всевозможных маршрутов, соединяющих вершины графа и выборе среди них маршрута с минимальной длиной. Для графа из \(N\) городов общее число возможных маршрутов равно \((N-1)!\), так как первый город фиксируется, а остальные \(N-1\) города могут быть упорядочены в любом порядке.
Этот метод проверяет все возможные комбинации маршрутов, поэтому при увеличении числа городов полное перечисление становится вычислительно неэффективным из-за экспоненциального роста числа возможных маршрутов.

\section{Муравьиный алгоритм}

Муравьиный алгоритм использует подход, основанный на поведении колонии муравьев. Каждый муравей ищет свой маршрут, опираясь на три основных фактора:

\begin{enumerate}
   \item \textbf{зрение} --- привлекательность перехода из города \(i\) в город \(j\), которая рассчитывается как:
   \begin{equation}\label{eq:vision}
   (\nu)_{ij} = \frac{1}{D_{ij}},
   \end{equation}
   где \(D_{ij}\) --- длина ребра \(i \rightarrow j\);

\item \textbf{память} --- множество уже посещенных муравьем \(k\) в день \(t\) городов \(J_k(t)\);

\item \textbf{обоняние} --- концентрация феромона \(\theta_{ij}(t)\) на ребре \(i \rightarrow j\) в день \(t\).
\end{enumerate}

На рассвете каждый муравей начинает маршрут из уникального города. Вероятность выбора города \(j\) из города \(i\) в день \(t\) рассчитывается следующим образом:
\begin{equation}\label{eq:probability}
P_{k,ij}(t) =
\begin{cases}
0, & \text{если } j \in J_k(t), \\
\frac{((\nu)_{ij})^\alpha \cdot (\theta_{ij}(t))^\beta}{\sum_{q \notin J_k(t)} ((\nu)_{iq})^\alpha \cdot (\theta_{iq}(t))^\beta}, & \text{если } j \notin J_k(t).
\end{cases}
\end{equation}
Параметры \(\alpha\) и \(\beta\) задают баланс между жадностью (\(\alpha\)) и стадностью (\(\beta\)) в решении.

После завершения всех маршрутов фаза ночи включает обновление феромонов на ребрах графа:
\begin{equation}\label{eq:pheromone_update}
\theta_{ij}(t+1) = \theta_{ij}(t) \cdot (1 --- \rho) + \Delta \theta_{ij}(t),
\end{equation}
где \(\rho \in (0, 1)\) --- коэффициент испарения, а \(\Delta \theta_{ij}(t)\) --- добавление феромонов, определяемое как:
\begin{equation}\label{eq:delta_pheromone}
\Delta \theta_{ij}(t) = \sum_{k} \Delta \theta_{k,ij}(t),
\end{equation}
\begin{equation}\label{eq:pheromone_individual}
\Delta \theta_{k,ij}(t) =
\begin{cases}
\frac{Q}{L_k(t)}, & \text{если муравей } k \text{ использовал ребро } i \rightarrow j, \\
0, & \text{иначе.}
\end{cases}
\end{equation}
Здесь \(Q\) --- дневная квота феромона, \(L_k(t)\) --- длина маршрута муравья \(k\).

\textbf{Модификация с элитными муравьями}

Элитные муравьи усиливают наиболее успешные маршруты, дополнительно увеличивая концентрацию феромонов на их ребрах. Это реализуется добавлением дополнительной квоты \(\Delta \theta_{elite,ij}(t)\), где лучшие маршруты выбираются на основе длины \(L_{best}\).

\section*{Вывод}

Была рассмотрена задача коммивояжера, которая из-за экспоненциального роста вычислительной сложности становится трудной для точного решения при большом числе городов. Метод полного перебора требует проверки всех возможных маршрутов, из-за чего невозможно его применение для графов с числом вершин более нескольких десятков.
Альтернативой является муравьиный алгоритм, который моделирует поведение колонии муравьев. Этот алгоритм адаптируется в процессе работы, усиливая феромонные следы на ребрах, принадлежащих наиболее перспективным маршрутам. Таким образом, муравьиный алгоритм позволяет находить решение задачи коммивояжера с меньшими вычислительными затратами по сравнению с методами полного перебора.

