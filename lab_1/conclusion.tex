\ssr{ЗАКЛЮЧЕНИЕ}

В данной работе была проведена оценка алгоритмов для вычисления расстояний Левенштейна и Дамерау-Левенштейна, с целью анализа их эффективности и времени выполнения. Были реализованы четыре алгоритма: 3 для поиска расстояния Левенштейна (итеративный, рекурсивный, рекурсивный с мемоизацией) и алгоритм поиска расстояния Дамерау-Левенштейна.

\vspace{0.5cm}
В ходе работы были выполнены следующие задачи: 
\begin{enumerate} 
\item Описана теоретическая основа алгоритмов. 
\item Реализованы алгоритмы поиска расстояний Левенштейна и Дамерау-Левенштейна. 
\item Проведено тестирование всех реализаций на различных строках, с целью оценки времени выполнения в зависимости от их длины. 
\item Замерено время выполнения для каждой реализации, и построен график зависимости времени от длины строк. 
\item Обоснован выбор языка программирования для реализации алгоритмов. 
\end{enumerate}

\vspace{0.5cm}
Результаты исследования показали, что итеративные алгоритмы имеют лучшее время выполнения по сравнению с рекурсивными. Использование мемоизации значительно улучшило производительность рекурсивного метода, хотя он все равно оказался менее эффективным по сравнению с итеративным подходом. 