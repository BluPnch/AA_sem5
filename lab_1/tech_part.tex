\chapter{Технологическая часть}

В данном разделе будут рассмотрены средства, использованные для разработки, а также приведены коды реализованных алгоритмов.

\section{Средства реализации}

Для реализации алгоритмов был выбран MicroPython, так как он обеспечивает эффективную работу на микроконтроллерах. Это позволило выполнить точные замеры времени выполнения.

\subsection{Итеративный алгоритм поиска расстояния Левенштейна}

\begin{center}
\begin{lstlisting}[caption={Итеративный алгоритм поиска расстояния Левенштейна}, label={lst:l_iteration}]
def l_iteration_matrix_cost(s1, s2):
    m, n = len(s1), len(s2)
    D = [[0] * (n + 1) for _ in range(m + 1)]

    for j in range(1, n + 1):
        D[0][j] = j
    for i in range(1, m + 1):
        D[i][0] = i

    for i in range(1, m + 1):
        for j in range(1, n + 1):
            D[i][j] = min(
                D[i - 1][j] + 1,
                D[i][j - 1] + 1,
                D[i - 1][j - 1] + is_replaced(s1[i - 1], s2[j - 1])
            )

    return D[m][n]
\end{lstlisting}
\end{center}

\subsection{Рекурсивный алгоритм поиска расстояния Левенштейна}

\begin{center}
\begin{lstlisting}[caption={Рекурсивный алгоритм поиска расстояния Левенштейна}, label={lst:l_recursion}]
def l_recursion(s1, s2, i, j):
    if i == 0:
        return j
    elif j == 0:
        return i
    else:
        return min(
            l_recursion(s1, s2, i - 1, j) + 1,
            l_recursion(s1, s2, i, j - 1) + 1,
            l_recursion(s1, s2, i - 1, j - 1) + is_replaced(s1[i - 1], s2[j - 1])
        )
\end{lstlisting}
\end{center}

\subsection{Рекурсивный алгоритм с мемоизацией}

\begin{center}
\begin{lstlisting}[caption={Рекурсивный алгоритм с мемоизацией для поиска расстояния Левенштейна}, label={lst:l_recursion_cache}]
def l_recursion_cache(s1, s2, i, j, memo):
    if memo[i][j] != -1:
        return memo[i][j]

    if i == 0:
        memo[i][j] = j
    elif j == 0:
        memo[i][j] = i
    else:
        memo[i][j] = min(
            l_recursion_cache(s1, s2, i - 1, j, memo) + 1,
            l_recursion_cache(s1, s2, i, j - 1, memo) + 1,
            l_recursion_cache(s1, s2, i - 1, j - 1, memo) + is_replaced(s1[i - 1], s2[j - 1])
        )
    return memo[i][j]
\end{lstlisting}
\end{center}

\subsection{Алгоритм Дамерау-Левенштейна}

\begin{center}
\begin{lstlisting}[caption={Алгоритм поиска расстояния Дамерау-Левенштейна}, label={lst:dl_matrix}]
def dl_matrix_cost(s1, s2):
    m, n = len(s1), len(s2)
    D = [[0] * (n + 1) for _ in range(m + 1)]

    for j in range(1, n + 1):
        D[0][j] = j
    for i in range(1, m + 1):
        D[i][0] = i

    for i in range(1, m + 1):
        for j in range(1, n + 1):
            D[i][j] = min(
                D[i - 1][j] + 1,
                D[i][j - 1] + 1,
                D[i - 1][j - 1] + is_replaced(s1[i - 1], s2[j - 1])
            )
            if i > 1 and j > 1 and s1[i - 1] == s2[j - 2] and s1[i - 2] == s2[j - 1]:
                D[i][j] = min(D[i][j], D[i - 2][j - 2] + 1)

    return D[m][n]
\end{lstlisting}
\end{center}

\section{Тестовые данные}

Для тестирования использовались строки различной длины и содержания, чтобы проверить корректность работы алгоритмов. Все тесты пройдены успешно. Тестовые данные представлены в таблице \ref{tab:test_data}.

\begin{center}
\begin{table}[H]
    \centering
    \caption{Тестовые данные для алгоритмов}
    \label{tab:test_data}
    \begin{tabular}{|c|c|c|c|}
        \hline
        Номер теста & Строка 1 & Строка 2 & Ожидаемое расстояние \\ \hline
        1 & "" & "" & 0 \\ \hline
        2 & "" & "a" & 1 \\ \hline
        3 & "" & "ab" & 2 \\ \hline
        4 & "a" & "" & 1 \\ \hline
        5 & "ab" & "" & 2 \\ \hline
        6 & "a" & "a" & 0 \\ \hline
        7 & "a" & "b" & 1 \\ \hline
        8 & "cat" & "cut" & 1 \\ \hline
        9 & "book" & "back" & 2 \\ \hline
        10 & "kitten" & "sitting" & 3 \\ \hline
        11 & "flaw" & "lawn" & 2 \\ \hline
        12 & "intention" & "execution" & 5 \\ \hline
    \end{tabular}
\end{table}
\end{center}

Все тесты пройдены успешно.

\section*{Вывод}

В данном разделе были рассмотрены алгоритмы вычисления расстояний Левенштейна и Дамерау-Левенштейна, а также их реализации на языке MicroPython. Программы успешно прошли тестирование на различных строках разной длины и содержания, что подтверждает их корректность и работоспособность.
