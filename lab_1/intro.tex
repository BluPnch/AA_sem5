\ssr{ВВЕДЕНИЕ}

В данной работе рассматриваются и сравниваются различные алгоритмы для вычисления расстояния Левенштейна и Дамерау-Левенштейна.
Целью данной лабораторной работы является анализ каждого из алгоритмов, их реализация и оценка эффективности с точки зрения времени выполнения. \cite{lit1}

\vspace{0.5cm}
В рамках исследования были реализованы следующие алгоритмы: \begin{itemize} 
\item итеративный алгоритм поиска расстояния Левенштейна, 
\item рекурсивный алгоритм поиска расстояния Левенштейна, 
\item рекурсивный алгоритм поиска расстояния Левенштейна с мемоизацией,
\item алгоритм Дамерау-Левенштейна. 
\end{itemize}


\vspace{0.5cm}
\textbf {Расстояние Левенштейна} - это минимальное количество односимвольных операций вставки, удаления и замены одного символа, необходимых для преобразования одной строки в другую.

\vspace{0.5cm}
Расстояние Левенштейна применяется в таких областях, как:
\begin{itemize} 
\item компьютерная лингвистика (исправление ошибок ввода),
\item биоинформатика.
\end{itemize}

\vspace{0.5cm}
\textbf {Задачи лабораторной работы}:
\begin{enumerate}
    \item Описать теоретическую основу алгоритмов.
    \item Реализовать алгоритмы поиска расстояний Левенштейна и Дамерау-Левенштейна.
    \item Провести тестирование всех реализаций на различных строках и оценить время выполнения в зависимости от длины строк.
    \item Замерить время выполнения для каждой реализации и построить график зависимости времени от длины строк.
    \item Обосновать выбор языка программирования.
\end{enumerate}

\clearpage
