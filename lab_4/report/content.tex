% Содержимое отчета по курсу Анализ алгоритмов

\aaunnumberedsection{ВВЕДЕНИЕ}{sec:intro}

В данном пункте уместно кратко описать некоторые вводные положения. Так, это один источник~\cite{etap_3}. 
А это два источника рядом~\cite{nlp_bolshakova_vorontsov, ronzhin_iface}. Автор данного текста желает 
читателям успехов.

Цель работы --- получение навыка организации параллельных вычислений по конвейерному принципу. 

Задачи работы: 
\begin{itemize}
    \item анализ предметной области;
    \item разработка алгоритма обработки данных;
    \item создание ПО, реализующего разработанный алгоритм;
    \item исследование характеристик созданного ПО.
\end{itemize}

\aasection{Входные и выходные данные}{sec:input-output}

Входными данными программы являются HTML-файлы. Каждый файл содержит одну страницу кулинарного рецепта, загруженную с 
веб-ресурса по варианту. Выходными данными является JSON-файл, содержащий извлеченную из входных файлов информацию, а именно: 
назание рецепта, список ингридиентов, список шагов выполнения рецепта. 

\aasection{Преобразование входных данных в выходные}{sec:algorithm}
 
Программа читает три числа из стандартного потока ввода, выполняет их умножение и выводит результат в стандартный поток вывода. 

\aasection{Примеры работы программы}{sec:demo}

На рисунке~\ref{img:demo} представлен пример работы программы (кликните по ссылке справа)~\cite{pizza_mishka}.

\FloatBarrier
\includeimage
{demo} % Имя файла без расширения (файл должен быть расположен в директории inc/img/)
{f} % Обтекание (без обтекания)
{h} % Положение рисунка (см. figure из пакета float)
{\textwidth} % Ширина рисунка
{Пример работы программы} % Подпись рисунка
\FloatBarrier


\aasection{Тестирование}{sec:tests}

В таблице~\ref{tbl:tests} представлены функциональные тесты для разработанного ПО. Все тесты пройдены успешно.

\begin{longtable}{|p{.2\textwidth - 2\tabcolsep}|p{.33\textwidth - 2\tabcolsep}|p{.24\textwidth - 2\tabcolsep}|p{.23\textwidth - 2\tabcolsep}|}
    \caption{Функциональные тесты}\label{tbl:tests} \\\hline
    № теста & Входные данные & Полученные выходные данные & Ожидаемые выходные данные                                          \\\hline
    \endfirsthead
    \caption{Функциональные тесты (продолжение)} \\\hline
    № теста & Входные данные & Полученные выходные данные  & Ожидаемые выходные данные                                                 \\\hline
    \endhead
    \endfoot
    1                                           & 1 2 3 & 6 & 6 \\\hline
    1                                           & 1 2 4 & 8 & 8 \\\hline
    1                                           & 1 5 -1 & -5 & -5 \\\hline
    \end{longtable}

\aasection{Описание исследования}{sec:study}

В ходе исследования требуется сформировать лог обработки задач. 
\newpage

В таблице~\ref{tbl:b_log} приведен фрагмент лога обработки. Обозначения событий:
\begin{itemize}
    \item start\_step\_1 — начло обработки на стадии 1;
    \item end\_step\_1 — окончание обработки на стадии 1;
    \item start\_step\_2 — начло обработки на стадии 2;
    \item end\_step\_2 — окончание обработки на стадии 2;
    \item start\_step\_3 — начло обработки на стадии 3;
    \item end\_step\_3 — окончание обработки на стадии 3.
\end{itemize}

\begin{longtable}{|p{.33\textwidth - 2\tabcolsep}|p{.33\textwidth - 2\tabcolsep}|p{.34\textwidth - 2\tabcolsep}|}
    \caption{Фрагмент лога обработки (начало)}\label{tbl:b_log}
    \\
    \hline
    Метка времени, мкс & Событие             & ID записи \\
    \hline
    \endfirsthead
    \caption{Фрагмент лога обработки (окончание)}
    \\
    \hline
    Метка времени, мкс   & Событие    & ID записи     \\
    \hline
    \endhead
    \hline
    \endfoot
    \endlastfoot
    \hline
    1666693286452510   & start\_step\_1  & 39068     \\ \hline
    1666693286452550   & end\_step\_1  & 39068     \\ \hline
    1666693286452560   & start\_step\_1 & 57060     \\ \hline
    1666693286452600   & start\_step\_1  & 57062     \\ \hline
    1666693286452640   & start\_step\_2    & 39068     \\ \hline
    1666693286452640   & start\_step\_1   & 61308     \\ \hline
    1666693286452660   & end\_step\_1 & 57060     \\ \hline
    1666693286452680   & end\_step\_2    & 39068     \\ \hline
    1666693286452680   & start\_step\_2  & 57060     \\ \hline
    1666693286452690   & end\_step\_1 & 57062     \\ \hline
    1666693286452690   & start\_step\_1    & 61309     \\ \hline
    1666693286452710   & start\_step\_2 & 57062     \\ \hline
    1666693286452710   & end\_step\_1    & 61308     \\ \hline
    1666693286452740   & start\_step\_2 & 61308     \\ \hline
    1666693286452740   & end\_step\_1    & 61309     \\ \hline
    1666693286452760   & start\_step\_2      & 61309     \\ \hline
    1666693286457930   & start\_step\_3   & 39068     \\ \hline
    1666693286457940   & end\_step\_3    & 39068     \\ \hline
    1666693286457950   & end\_step\_2      & 57060     \\ \hline
    1666693286461440   & start\_step\_3   & 57060     \\ \hline
    1666693286461450   & end\_step\_3    & 57060     \\ \hline
    1666693286461460   & end\_step\_2      & 57062     \\ \hline
    1666693286465340   & start\_step\_3   & 57062     \\ \hline
    1666693286465350   & end\_step\_3    & 57062     \\ \hline
    1666693286465360   & end\_step\_2      & 61308     \\ \hline
    1666693286468670   & start\_step\_3   & 61308     \\ \hline
    1666693286468680   & end\_step\_3    & 61308     \\ \hline
    1666693286468690   & end\_step\_2      & 61309     \\ \hline
    1666693286471980   & start\_step\_3      & 61309     \\ \hline
    1666693286471990   & end\_step\_3      & 61309     \\ \hline
\end{longtable}

По результатам проеденного исследования сделан вывод о том, что события лога упорядочены по возрастанию временных меток. 

\aaunnumberedsection{ЗАКЛЮЧЕНИЕ}{sec:outro}

Цель работы достигнута. Решены все поставленные задачи: 
\begin{itemize}
    \item анализ предметной области;
    \item разработка алгоритма обработки данных;
    \item создание ПО, реализующего разработанный алгоритм;
    \item исследование характеристик созданного ПО.
\end{itemize}
